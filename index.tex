% Options for packages loaded elsewhere
\PassOptionsToPackage{unicode}{hyperref}
\PassOptionsToPackage{hyphens}{url}
%
\documentclass[
]{article}
\usepackage{amsmath,amssymb}
\usepackage{lmodern}
\usepackage{iftex}
\ifPDFTeX
  \usepackage[T1]{fontenc}
  \usepackage[utf8]{inputenc}
  \usepackage{textcomp} % provide euro and other symbols
\else % if luatex or xetex
  \usepackage{unicode-math}
  \defaultfontfeatures{Scale=MatchLowercase}
  \defaultfontfeatures[\rmfamily]{Ligatures=TeX,Scale=1}
\fi
% Use upquote if available, for straight quotes in verbatim environments
\IfFileExists{upquote.sty}{\usepackage{upquote}}{}
\IfFileExists{microtype.sty}{% use microtype if available
  \usepackage[]{microtype}
  \UseMicrotypeSet[protrusion]{basicmath} % disable protrusion for tt fonts
}{}
\makeatletter
\@ifundefined{KOMAClassName}{% if non-KOMA class
  \IfFileExists{parskip.sty}{%
    \usepackage{parskip}
  }{% else
    \setlength{\parindent}{0pt}
    \setlength{\parskip}{6pt plus 2pt minus 1pt}}
}{% if KOMA class
  \KOMAoptions{parskip=half}}
\makeatother
\usepackage{xcolor}
\usepackage[margin=1in]{geometry}
\usepackage{graphicx}
\makeatletter
\def\maxwidth{\ifdim\Gin@nat@width>\linewidth\linewidth\else\Gin@nat@width\fi}
\def\maxheight{\ifdim\Gin@nat@height>\textheight\textheight\else\Gin@nat@height\fi}
\makeatother
% Scale images if necessary, so that they will not overflow the page
% margins by default, and it is still possible to overwrite the defaults
% using explicit options in \includegraphics[width, height, ...]{}
\setkeys{Gin}{width=\maxwidth,height=\maxheight,keepaspectratio}
% Set default figure placement to htbp
\makeatletter
\def\fps@figure{htbp}
\makeatother
\setlength{\emergencystretch}{3em} % prevent overfull lines
\providecommand{\tightlist}{%
  \setlength{\itemsep}{0pt}\setlength{\parskip}{0pt}}
\setcounter{secnumdepth}{-\maxdimen} % remove section numbering
\ifLuaTeX
  \usepackage{selnolig}  % disable illegal ligatures
\fi
\IfFileExists{bookmark.sty}{\usepackage{bookmark}}{\usepackage{hyperref}}
\IfFileExists{xurl.sty}{\usepackage{xurl}}{} % add URL line breaks if available
\urlstyle{same} % disable monospaced font for URLs
\hypersetup{
  pdftitle={Welcome to the R Crash Course!},
  pdfauthor={A beginners guide to R from downloading R and RStuido to formatting and visualizing data},
  hidelinks,
  pdfcreator={LaTeX via pandoc}}

\title{Welcome to the R Crash Course!}
\author{A beginners guide to R from downloading R and RStuido to
formatting and visualizing data}
\date{June 2023}

\begin{document}
\maketitle

\begin{itemize}
\tightlist
\item
  \textbf{When?} Tuesday June 27th to Thursday June 29, 2023: 9am - 6pm
\item
  \textbf{Where?} Casa Tisaru Lepsa, Romania
\item
  \textbf{What to prepare?} Please bring your laptop computer! If
  possible, install R and RStudio prior to the course (see
  \href{Links.html}{links}).
\end{itemize}

\hypertarget{facilitator}{%
\subsection{Facilitator}\label{facilitator}}

\href{https://github.com/marissadyck}{Marissa Dyck}\\
Ohio University\\
Department of Biological Sciences\\
Email: \url{marissadyck17@gmail.com}

\hypertarget{contributors}{%
\subsection{Contributors}\label{contributors}}

This workshop is based on content from an R Bootcamp run by Dr.~Kevin
Shoemaker at University of Nevada Reno
(\url{https://github.com/kevintshoemaker/R-Bootcamp})

\href{http://naes.unr.edu/shoemaker/}{Kevin Shoemaker}\\
University of Nevada Reno\\
Department of Natural Resources and Environmental Science\\
Email: \url{kshoemaker@cabnr.unr.edu}

\hypertarget{about}{%
\subsection{About}\label{about}}

The statistical programming software `R' is one of the fundamental tools
for modern data exploration and is a useful tool for data processing,
statistical analysis, and production of high-quality figures.

This workshop is designed for beginner to intermediate R users and will
focus on increasing comfort and familiarity with using R and RStudio, R
syntax, and troubleshooting code. We will begin with the basics (what is
R? How to install R and RStudio? Navigating and understanding the layout
of Rstudio) and continue onto data manipulation and formatting,
visualizing data, working with packages, and troubleshooting code. The
main goal of the workshop is to ensure participants have enough
proficiency and confidence with data operations and programming in R to
engage in productive, self-directed learning and problem-solving.

All code will be available as scripts that you can download from this
website (at the top of each module page on this website) and load up in
RStudio. That way you won't need to constantly copy and paste from the
web!

\hypertarget{before-we-get-started}{%
\subsection{Before we get started\ldots{}}\label{before-we-get-started}}

Before we dig in and get started with the modules, you should install R
and RStudio. Even if you have installed R and RStudio before you should
ensure you have the latest version, if you aren't sure how to check use
the links to install the programs again just to be sure and we will
cover how to check your version during the workshop. Here are some links
to help you get started:

\href{https://cran.r-project.org/}{Download and install R}\\
\href{https://www.rstudio.com/products/rstudio/download/}{Download and
install RStudio} (use free version!)

Also, it can be very helpful to print out R `cheatsheets' and bring that
with you (we will also have some available at the workshop!). Here are
some links:

\href{https://www.rstudio.com/wp-content/uploads/2016/05/base-r.pdf}{Base
R cheatsheet}\\
\href{https://cran.r-project.org/doc/contrib/Short-refcard.pdf}{R
reference card}\\
\href{https://posit.co/resources/cheatsheets/?type=posit-cheatsheets\&_page=2/}{Various
R cheatsheets} (I recommend `Data tidying with tidyr', `Data
transofrmation with dpylr', and `Data visualization with ggplot2'
cheatsheets for this workshop)

Okay, now we're ready to go!

\href{day1_1.html}{--go to first module--}

\end{document}
